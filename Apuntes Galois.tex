\documentclass[10pt, a4paper]{article}

\usepackage[utf8]{inputenc}
\usepackage{amsfonts}
\usepackage{amsmath}
\usepackage{amssymb}
\usepackage{vmargin}
\usepackage{mathtools}
\usepackage{enumitem}
\usepackage{cancel}
\usepackage{hyperref}
\usepackage{tikz}
\usepackage{faktor}
\usepackage{tcolorbox}
\tcbuselibrary{theorems}

% Comandos útiles
\def\checkmark{\tikz\fill[scale=0.4](0,.35) -- (.25,0) -- (1,.7) -- (.25,.15) -- cycle;}
\newcommand{\R}{\mathbb{R}}
\newcommand{\N}{\mathbb{N}}
\newcommand{\Z}{\mathbb{Z}}
\newcommand{\C}{\mathbb{C}}
\newcommand{\Q}{\mathbb{Q}}
\newcommand{\F}{\mathbb{F}}
\newcommand{\obs}{\underline{Observación:} }
\newcommand{\ej}{\underline{Ejemplo:} }
\newcommand{\ejs}{\underline{Ejemplos:} }
\newcommand{\nota}{\underline{Notación:} }
\newcommand{\demo}{\underline{Demostración:} }
\newcommand{\anillo}[1][]{\hyperref[def:anillo]{anillo}#1 }
\newcommand{\cuerpo}[1][]{\hyperref[def:cuerpo]{cuerpo}#1 }

\newenvironment{enumerater}{\begin{enumerate}[label=\roman*)]}
{\end{enumerate}}
\newenvironment{enumeratea}{\begin{enumerate}[label=\arabic*)]}
{\end{enumerate}}

\tcbset{
	% Definición de estilos de recuadros
	defstyle/.style={colback=blue!5, colframe=blue!70!black, before={\vspace{5mm}}, after={\vspace{5mm}}}, % Definiciones
	theostyle/.style={colback=red!5, colframe=red!80!black, before={\vspace{5mm}}, after={\vspace{5mm}}}, % Teoremas
	propstyle/.style={colback=green!5, colframe=green!60!black, before={\vspace{5mm}}, after={\vspace{5mm}}} % Proposiciones
}

% Definicion de las cajas
\newtcbtheorem[number within=section]{definition}{Definición}{defstyle}{def}
\newtcbtheorem[number within=section]{theorem}{Teorema}{theostyle}{theo}
\newtcbtheorem[number within=section]{proposition}{Proposición}{propstyle}{prop}

\setlength{\parindent}{0pt}
\setmargins{2.5cm}       % margen izquierdo
{1.5cm}                        % margen superior
{16cm}                      % anchura del texto
{23.42cm}                    % altura del texto
{10pt}                           % altura de los encabezados
{1cm}                           % espacio entre el texto y los encabezados
{0pt}                             % altura del pie de página
{2cm}

\begin{document}

\title{Teoría de Galois}
\author{Carlos Gómez-Lobo}
\date{}
\maketitle

\section{Anillos}

A continuación vamos a repasar algunos conceptos sobre anillos y especialmente anillos de polinomios, empezando por la definición de anillo.

\begin{definition}{Anillo}{anillo}
Un \textbf{anillo} es un conjunto no vacío dotado de dos operaciones, que denotaremos como suma (+) y multiplicación ($\cdot$) y que cumplen las siguientes propiedades:
\begin{itemize}
	\item $(R, +)$: grupo abeliano
	\item $(R, \cdot)$: operación binaria interna y cumple la propiedad asociativa
\end{itemize}
Si además $(R, \cdot)$ tiene identidad, es decir, existe un elemento $e \in R$ tal que $e \cdot r = r \cdot e = r \; \forall r \in R$, diremos que $R$ es un anillo con unidad y si además es abeliano, entonces será un anillo conmutativo.
\end{definition}

A nosotros en esta asignatura nos interesarán especialmente estos últimos y nos referiremos a estos simplemente como anillos sin especificar que son conmutativos y sin unidad.

\vspace{3mm}

\underline{Ejemplos:} $\Z, \Z_n, \R, \C, \Q, M_n(\R)$ (no conmutativo), etc.

\vspace{3mm}

\underline{Notación:} \begin{itemize}
	\item 0 para el elemento neutro de la suma
	\item $-a$ para el elemento inverso aditivo (opuesto).
	\item 1 para el elemento neutro de la multiplicación
	\item $a^{-1}$ para el inverso multiplicativo, si existe
	\item $na = \underbrace{a + ... + a}_{\text{n veces}}$
	
	\vspace{-3mm}
	
	\item $a^n = \underbrace{a \cdot ... \cdot a}_{\text{n veces}}$
\end{itemize}

\begin{definition}{Cuerpo}{cuerpo}
Un anillo $(R, +, -)$ es un \textbf{cuerpo} si $(R^* = R \backslash \{0\}, \cdot)$ es un grupo abeliano.
\end{definition}

\underline{Ejemplos:} $\Q, \R, \C, \Z_p$ p primo, etc.

\vspace{3mm}

\underline{Notación:} $\; \Z_n  \left \{
\begin{matrix*}[l]
\textrm{grupo aditivo} \rightarrow \C_n \\
\textrm{anillo} \rightarrow \Z_n \\
\textrm{cuerpo} \rightarrow \mathbb{F}_n \textrm{(n primo)}
\end{matrix*} \right .$

\begin{definition}{Divisor de cero}{div_cero}
Sea $R$ un anillo. Diremos que un elemento $a \in R, \; a \neq 0$ es un \textbf{divisor de cero} si $\exists b \in R, \; b \neq 0$ tal que $a \cdot b = 0$.
\end{definition}

\ej En $\Z_6: \bar{2}, \bar{3} \neq \bar{0}$ y $\bar{2} \cdot \bar{3} = 0$.

\begin{definition}{Dominio de integridad}{DI}
Sea $R$ un anillo, si $R$ no tiene divisores de cero, entonces se dice que es un \textbf{dominio de integridad}.
\end{definition}

\ejs $\Z, \Q, \R, \C, \F_p$

\begin{definition}{"Divide a"}{dividea}
Diremos que $a$ \textbf{divide a} $b$ en $R$ si $\exists c \in R$ tal que $b = a \cdot c$ y escribiremos $a | b$.
\end{definition}

\subsection{Subanillos}

\vspace{3mm}

\begin{definition}{Subanillo}{subanillo}
Diremos que $S \subset R$ es un \textbf{subanillo} si $(S, +, \cdot)$ es un anillo.
\end{definition}

\obs $S \subset R$ es un subanillo s y solo si:

\begin{enumerate}[label=\arabic*)]
	\item $S \neq \emptyset$
	\item $\forall a, b \in S, a + b \in S$
	\item $\forall a, b \in S, a \cdot b \in S$
	\item $1 \in S$
\end{enumerate}

\vspace{3mm}

\ej $\Z \subset \Q \subset \R \subset \C$

\begin{definition}{Menor subanillo que contiene a un elemento}{menoranillo}
Dado un anillo $R$ y un elemento $a$, podemos definir el \textbf{menor subanillo que contiene a $\mathbf{R}$ y al elemento $\mathbf{a}$} como $R[a] = \left \{ \displaystyle\sum r_i \cdot a^k, \forall r \in R; i, k \in \N \right \}$
\end{definition}

\ej $\Z[i] = \{a + bi, a, b \in \Z\} \subset \C$. Otra forma de ver este anillo es como la intersección de todos los subanillos de $\C$ que contienen a $\Z$ y a $i$.

\vspace{3mm}

\obs De la misma forma podemos definir el menor cuerpo que contiene a un elemento y que denotamos como $R(a)$.

\vspace{3mm}

\ej $\Q[\sqrt{2}] = \{a + b\sqrt{2}, \; a, b \in \Q\}$, $\Q(\sqrt{2}) = \bigg \{ \dfrac{a + b\sqrt{2}}{\underbrace{c + d\sqrt{2}}_{\neq 0}}, \; a, b, c, d \in \Q \bigg \}$, $\Q \subset \Q[\sqrt{2}] \subset \Q(\sqrt{2}) \subset \R$

\subsection{Anillos de polinomios}

\vspace{3mm}

\begin{definition}{Anillo de polinomios}{anillopoli}
Sea $R$ un anillo, llamaremos a $R[x]$ al \textbf{anillo de polinomios con coeficientes en $\mathbf{R}$} y que será de la forma $R[x] = \left \{ \displaystyle\sum_{k = 0}^{n} r_k x^k, \; \forall r \in R \right \}$.
\end{definition}

\ejs $\C[x], \R[x], \Q[x], \Z[x]$, etc.

\begin{definition}{Coeficiente director}{coef_dir}
El \textbf{coeficiente director} de un polinomio es el coeficiente distinto de 0 que multiplica a la x de mayor grado.
\end{definition}

\nota Grado de $p(x) := deg(p(x))$

\vspace{3mm}

\begin{proposition}{}{grado_prod}
El grado del producto de dos polinomios puede tener distintos valores en función de si el anillo sobre el que se construye es o no un \hyperref[def:DI]{DI}:
\[
deg(p(x) \cdot q(x)) = \left \{
\begin{matrix*}[l]
deg(p(x)) + deg(q(x)) \text{ si $R$ \hyperref[def:DI]{es dominio de integridad}} \\
\leq deg(p(x)) + deg(q(x)) \text{ si no lo es}
\end{matrix*} \right .
\]
\end{proposition}

\demo Obvio.

\vspace{3mm}

\ej $\Z_4, \left .
\begin{matrix*}[l]
p(x) = 2x + 1 \\
q(x) = 2x
\end{matrix*} \right \}
deg(p(x) \cdot q(x) = 1 < 2$

\vspace{3mm}

\begin{proposition}{}{grado_DI}
 Sea $R$ un \hyperref[def:cuerpo]{cuerpo}, entonces $R$ es siempre \hyperref[def:DI]{dominio de integridad} y para cualesquiera polinomios de $R[x]$ se cumple que $deg(p(x)) \cdot deg(q(x)) = deg(p(x)) + deg(q(x))$.	
\end{proposition} 
 
\demo Para demostrar que un cuerpo siempre es un \hyperref[def:DI]{DI} vamos a ver por reducción al absurdo que todo elemento de un anillo que tenga inverso multiplicativo no es \hyperref[def:div_cero]{divisor de cero}.

Suponemos que $r \neq 0 \in R$ es \hyperref[def:div_cero]{divisor de cero}, es decir, $\exists r^{-1}$ tal que $r' \neq 0, r \cdot r' = 0$. Ahora suponemos además que $r$ es invertible, es decir, $\exists r^{-1}$ tal que $r \cdot r^{-1} = 1$. Entonces $r \cdot r^{-1} = 1 \implies (r' \cdot r) \cdot r^{-1} = b \implies 0 = b$. Contradicción.

De la misma forma se puede ver que una unidad no puede ser un \hyperref[def:div_cero]{divisor de cero} y como en un cuerpo todos sus elementos son unidades, no hay ningún \hyperref[def:div_cero]{divisor de cero} y por tanto es un \hyperref[def:DI]{dominio de integridad}.

Por esto y por la proposición \ref{prop:grado_prod}, queda demostrado.

\begin{proposition}{}{apoli_nocuerpo}
Sea $K$ un cuerpo, entonces el anillo de polinomios asociado a $K, \; K[x]$  \textbf{no} es un cuerpo y sus únicos elementos invertibles son los pertecientes al cuerpo $K$ no nulos.
\end{proposition}

\demo Sea $p(x) \in K[x], \; p(x) \neq 0$ invertible en $K[x]$. Entonces $p(x) \cdot p^{-1}(x) = 1, \; deg(1) = 0$ y como por la proposición \ref{prop:grado_prod}, $deg(p(x) \cdot p^{-1}(x)) \leq deg(p(x)) + deg(p^{-1}(x))$, se tiene que $deg(p(x)) = deg(p^{-1}(x)) = 0$, por lo que los únicos elementos invertibles en $K[x]$ son los de grado 0, que son los no nulos que pertenecen a $K$. Entonces, puesto que no todos los elementos de $K[x]$ son invertibles, $K[x]$ no es un cuerpo.

\vspace{3mm}

\obs El menor cuerpo que contiene a x y a los elementos de $K$ es \\ $K(x) = \left \{ \dfrac{p(x)}{q(x)} : p(x), q(x) \in K[x], q(x) \neq 0 \right \}$, teniendo en cuenta que $\dfrac{p(x)}{q(x)} = \dfrac{p'(x)}{q'(x)} \iff p(x) q'(x) = p'(x) q(x)$.

\begin{definition}{Polinomio mónico}{poli_mon}
Un \textbf{polinomio mónico} es aquel cuyo \hyperref[def:coef_dir]{coeficiente director} es 1.
\end{definition}

\subsection{Ideales en un anillo}

\vspace{3mm}

\begin{definition}{Ideal}{ideal}
Sea $R$ un anillo. Un \textbf{ideal} en $R$ es un subconjunto no vacío $I \subset R$ tal que:
\begin{enumerate}[label=\roman*)]
	\item $(I, +)$ es un subgrupo de $R$.
	\item $\forall r \in R, \; \forall a \in I, \; r \cdot a \in I$ (Propiedad de absorción).
\end{enumerate}
\end{definition}

\begin{proposition}{Criterio para ideales}{crit_ideal}
Para que un subanillo $I \subset R, \; I \neq \emptyset$ sea un ideal tiene que cumplir que:

\begin{enumerate}[label=\roman*)]
	\item $\forall a, b \in I, \; a - b \in I \; (a + b \in I)$.
	\item $\forall r \in R, \; \forall a \in I, \; r \cdot a \in I$.
\end{enumerate}
\end{proposition}

\ejs
\begin{enumerate}[label=\arabic*)]
	\item $R$ anillo cualquiera
		\begin{enumerate}[label=\roman*)]
			\item $R$ es un ideal (el ideal trivial).
			\item \{0\} siempre es un ideal.
		\end{enumerate}

Si $I \subset R$ es un ideal e $I \neq R$, diremos que $I$ es un \textbf{ideal propio}.

	\item En $\Z$ todos los anillos de la forma $I = \{2n : n \in \Z\}$ son ideales.
	
	\item $\Q[x], \; I = \{p(x) : p(r_0) = 0, \; r_0 \in \Q \}$
			
			Comprobación: Sean $p(x), q(x), t(x) \in \Q[x]$ tal que $p(r_0) = q(r_0) = 0,  \; t(x)$ cualquiera, entonces:
		\begin{enumerater}
			\item $s(r_0) = p(r_0) - q(r_0) = 0 \implies s(x) \in I$.
			\item $z(r_0) = p(r_0) \cdot t(r_0) = 0 \implies z(x) \in I$.
		\end{enumerater}
	\item 
\end{enumerate}

\begin{proposition}{}{ideales_z}
Todos los ideales de $\Z$ son de la forma $\{kn : n \in \Z\}$.
\end{proposition}

\demo Sale del algoritmo de la división.

\vspace{5mm}

\obs Sea $R$ un anillo y sean $I, J \subset R$ ideales, entonces:

\begin{enumerater}
	\item En general, $I \cup J$ \textbf{no} es un ideal.
	\item $I \cap J$ es un ideal
\end{enumerater}

\begin{proposition}{}{ideales_cuerpo}
Sea $K$ un anillo, entonces $K$ es un \hyperref[def:cuerpo]{cuerpo} si y solo si continene dos \hyperref[def:ideal]{ideales}: $\{0\}$ y $K$.
\end{proposition}

\demo 

$\implies$) Sea $I \in K, \; I \neq \{0\}$ un ideal y $r \in I, \; r \neq 0$ uno de sus elementos. Por ser $K$ un cuerpo $\exists r^{-1}$ tal que $r \cdot r^{-1} = 1 \in I$ (Propiedad de absorción) $\implies I = K$.

$\impliedby$) Sea $K$ un anillo y $r \in K, \; r \neq 0$. Vamos a ver que $r$ tiene un inverso.

Definimos $I := \{rs : s \in K\}$ que es un ideal. Puesto que $I \neq \{0\}$ y solo hay dos ideales, $I = K \implies 1 \in K \implies \exists s \in K$ tal que $s \cdot r = 1 \implies s = r^{-1}$.

\begin{definition}{Ideal generado}{ideal_generado}
Sea $R$ un \anillo y $\{r_i\}$ una familia de elementos de $R$. Diremos que el \textbf{ideal generado} por $\{r_i\}_{i \in I}$ es el ideal más pequeño que contiene a $\{r_i\}_{i \in I}$ y lo denotamos por $<r_i>_{i \in I} = \left \{ \displaystyle\sum s_j r_i : s_j \in R \right \}$.
\end{definition}

\ej En $\Z[x]$ el ideal generado por $<2, x> = \{2q(x) + xp(x)\ : q(x), p(x) \in \Z\}$

\begin{definition}{Ideal principal}{ideal_principal}
Sea $R$ un \anillo, diremos que $I \subset R$ es un \textbf{ideal principal} si $\exists a \in R$ tal que $I = <a>$.
\end{definition}

\ej
\begin{enumeratea}
\item En $\Z$ todos los ideales son principales.
\item $<2, x> \subset \Z[x]$ no es principal.

Comprobación: Suponemos que $\exists g(x) \in \Z[x]$ tal que $<2, x> = <g(x)>$, entonces $\exists q(x)$ tal que $g(x) \cdot q(x) = 2 \implies deg(g(x)) = 0 \implies g(x) = k \in \Z \implies k = \pm 1, \pm 2$

Supongamos que $k = \pm 1$. Entonces $<g(x)> = <\pm 1> =<2, x> = \Z[x]$. Sin embargo, $1 = \underbrace{2p(x)}_{\text{coef. par}} + \cancelto{0}{\underbrace{q(x)x}_{grado \geq 1}}$. Contradicción.

Ahora si suponemos que $k = \pm 2 \implies <g(x)> = <\pm 2> = <2, x> =$ polinomios con coeficientes pares, pero $x \notin <\pm 2>$. Contradicción.
\end{enumeratea}

\begin{definition}{Dominio de ideales principales (DIP)}{DIP}
Sea $R$ un \anillo{,} si todos los \hyperref[def:ideal]{ideales} contenidos en $R$ con principales se dice que es un \textbf{dominio de ideales principales}.
\end{definition}

\begin{proposition}{}{cuerpo_ideal_ppal}
Sea $K$ un \cuerpo entonces $K[x]$ es un \hyperref[def:DIP]{dominio de ideales principales}.
\end{proposition}

\demo Sea $I \subset K[x]$ un ideal.
\begin{itemize}
\item Si $I = \{0\}$ \checkmark
\item Suponemos que $I \neq \{0\} \implies \exists p(x) \in I, \; p(x) \neq 0$ y podemos definir $\Lambda = \{deg(p(x)) : p(x) \in I\} \neq \emptyset, \; \Lambda \subset \N$. Por la propiedad de buen orden de $\N$ podemos afirmar que $\Lambda$ tiene un elemento mínimo $n$, por lo que $\exists p(x) \in I$ tal que $deg(px) = n$ y además $<p(x)> \subseteq I$. Ahora vamos a demostrar por el algoritmo de la división de polinomios que $<p(x)> = I$.

Sea $s(x) \in I \implies s(x) = q(x)p(x) + r(x)$ y hay dos posibilidades para $r(x)$:
	\begin{itemize}[label=$\circ$]
		\item $r(x) = 0 \implies p(x) \mid q(x) \; \checkmark$
		\item $r(x) \neq 0, \; \underbrace{s(x)}_{\in I} = \underbrace{q(x) \underbrace{p(x)}_{\in I}}_{\in I} + r(x) \overset{Prop. 1}{\implies} r(x) \in I$. Contradicción porque $deg(r(x)) < deg(p(x))$ que es el grado mínimo en $I$.
	\end{itemize}
\end{itemize}

\ej Usando un argumento similar con el algoritmo de la división en $\Z$ se puede probar que este es un \hyperref[def:DIP]{DIP}.

\vspace{5mm}

\obs El generador de un ideal $I \subset K[x]$ no tiene por qué ser único: si $I = <p (x)>$ y $a \in K$, entonces $I = <ap(x)>$. Para describir estos anillos de forma canónica utilizaremos como generador un \hyperref[def:poli_mon]{polinomio mónico}.

\subsection{Anillos cociente}

\vspace{3mm}

\begin{definition}{Anillo conciente}{anillo_conciente}
Sea $I \subset R$ un ideal en $R$, podemos definir como en los grupos al conjunto $\faktor{R}{I}$ como el \textbf{anillo cociente} según la relación de equivalencia $a = b \iff a - b \in I$.
\end{definition}

Ahora vamos a comprobar algunas cosas sobre la definición anterior:

\begin{enumerate}
	\item La relación de equivalencia usada es realmente una relación de equivalencia estudiando sus tres propiedades:
	\begin{enumerater}
		\item Reflexiva: $a - a = 0 \in I$ \checkmark
		\item Simétrica: $a = b \implies a - b \in I \implies (a - b) \cdot -1 \in I \implies (b - a) \in I \implies b = a$ \checkmark
		\item Transitiva: $a = b$ y $b = c \implies a - b \in I$ y $b - c \in I \overset{\text{Prop. 1}}{\implies} a - b + b -c = a - b \in I \implies a = c \;$\checkmark
	\end{enumerater}
	\item El conjunto cociente resultado tiene estructura de anillo. Para ello solo es necesario comprobar que el producto está bien definido, es decir, de dos elementos no depende del representante escogido.
	
	Sean $\bar{a} = \{a + I\}, \bar{b} = \{b + I\}$, entonces $(a + I)(b + I) = ab + \overbrace{\underbrace{aI}_{\in I} + \underbrace{bI}_{\in I} + I}^{\in I} = ab + I \implies \bar{a}\bar{b} = \overline{ab} \;$\checkmark
\end{enumerate}

\vspace{3mm}

\obs 
\begin{enumerate}
	\item Si el anillo $R$ es conmutativo y con unidad, entonces $\faktor{R}{I}$ tamibién lo es y su unidad es $\bar{1}$.
	\item $\forall a \in I, \; \bar{c} = 0$.
\end{enumerate}

\ejs 
\begin{enumeratea}
	\item $\faktor{\Z}{n\Z} = \Z_n$
	\item $\faktor{R}{R} = \{0\}$
	\item $\faktor{R}{\{0\}} = R$
	\item $S = \faktor{\R[x] \;}{<x^2 + 1>}$: ¿Qué pinta tiene? En primer lugar, vamos a comprobar que todo elemento de $S$ es equivalente a un elemento de la forma $ax + b, \; a, b \in \R$.
	
	Sea $p(x) \in \R[x]$, ¿$\overline{p(x)}$? $p(x) = q(x)(x^2 + 1) + r(x)$ donde $r(x) = 0$ ó $deg(r(x)) \leq 1 \implies p(x) - r(x) \in <x^2 + 1> \implies \overline{p(x)} = \overline{r(x)}$.
\end{enumeratea}

\end{document}