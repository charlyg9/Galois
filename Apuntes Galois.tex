\documentclass[10pt, a4paper]{article}
\usepackage[utf8]{inputenc}
\usepackage{amsfonts}
\usepackage{amsmath}
\usepackage{tcolorbox}
\usepackage{vmargin}
\usepackage{mathtools}
\usepackage{enumitem}
\tcbuselibrary{theorems}

\newcommand{\R}{\mathbb{R}}
\newcommand{\N}{\mathbb{N}}
\newcommand{\Z}{\mathbb{Z}}
\newcommand{\C}{\mathbb{C}}
\newcommand{\Q}{\mathbb{Q}}
\newcommand{\F}{\mathbb{F}}
\newcommand{\obs}[1][\!\!]{ \underline{Observación #1:} }
\newcommand{\ej}{\underline{Ejemplo:} }
\newcommand{\ejs}{\underline{Ejemplos:} }

\tcbset{
	defstyle/.style={colback=blue!5, colframe=blue!60!black, before={\vspace{5mm}}, after={\vspace{5mm}}},
	theostyle/.style={colback=red!5, colframe=red!60!black, before={\vspace{5mm}}, after={\vspace{5mm}}}
}

\newtcbtheorem[number within=section]{definition}{Definición}{defstyle}{def}
\newtcbtheorem[number within=section]{theorem}{Teorema}{theostyle}{theo}

\setlength{\parindent}{0pt}
\setmargins{2.5cm}       % margen izquierdo
{1.5cm}                        % margen superior
{16cm}                      % anchura del texto
{23.42cm}                    % altura del texto
{10pt}                           % altura de los encabezados
{1cm}                           % espacio entre el texto y los encabezados
{0pt}                             % altura del pie de página
{2cm}

\begin{document}

\title{Teoría de Galois}
\author{Carlos Gómez-Lobo}
\maketitle

\begin{abstract}
El objetivo de la asignatura será estudiar toda la teoría necesaria para demostrar el teorema de Galois.
\end{abstract}
\newpage

\section{Repaso de anillos}

A continuación vamos a repasar algunos conceptos sobre anillos y especialmente anillos de polinomios, empezando por la definición de anillo.

\begin{definition}{Anillo}{anillo}
Un \textbf{anillo} es un conjunto no vacío dotado de dos operaciones, que denotaremos como suma (=) y multiplicación (*) y que cumplen las siguientes propiedades:
\begin{itemize}
	\item $(R, +)$: grupo abeliano
	\item $(R, \cdot)$: operación binaria interna y cumple la propiedad asociativa
\end{itemize}
Si además $(R, \cdot)$ tiene identidad, es decir, existe un elemento $e \in R$ tal que $e \cdot r = r \cdot e = r \; \forall r \in R$, diremos que $R$ es un anillo con unidad y si además es abeliano, entonces será un anillo conmutativo.
\end{definition}

A nosotros en esta asignatura nos interesarán especialmente estos últimos y nos referiremos a estos simplemente como anillos sin especificar que son conmutativos y sin unidad.

\vspace{3mm}

\underline{Ejemplos:} $\Z, \Z_n, \R, \C, \Q, M_n(\R)$ (no conmutativo), etc.

\vspace{3mm}

\underline{Notación:} \begin{itemize}
	\item 0 para el elemento neutro de la suma
	\item $-a$ para el elemento inverso aditivo (opuesto).
	\item 1 para el elemento neutro de la multiplicación
	\item $a^{-1}$ para el inverso multiplicativo, si existe
	\item $na = \underbrace{a + ... + a}_{n veces}$
	
	\vspace{-3mm}
	
	\item $a^n = \underbrace{a \cdot ... \cdot a}_{n veces}$
\end{itemize}

\begin{definition}{Cuerpo}{cuerpo}
Un anillo $(R, +, -)$ es un \textbf{cuerpo} si $(R^* = R \backslash \{0\}, \cdot)$ es un grupo abeliano.
\end{definition}

\underline{Ejemplos:} $\Q, \R, \C, \Z_p$ p primo, etc.

\vspace{3mm}

\underline{Notación:} $\; \Z_n  \left \{
\begin{matrix*}[l]
\textrm{grupo aditivo} \rightarrow \C_n \\
\textrm{anillo} \rightarrow \Z_n \\
\textrm{cuerpo} \rightarrow \mathbb{F}_n \textrm{(n primo)}
\end{matrix*} \right .$

\begin{definition}{Divisor de cero}{div_cero}
Sea $R$ un anillo. Diremos que un elemento $a \in R, \; a \neq 0$ es un \textbf{divisor de cero} si $\exists b \in R, \; b \neq 0$ tal que $a \cdot b = 0$.
\end{definition}

\ej En $\Z_6: \bar{2}, \bar{3} \neq \bar{0}$ y $\bar{2} \cdot \bar{3} = 0$.

\begin{definition}{Dominio de integridad}{DI}
Sea $R$ un anillo, si $R$ no tiene divisores de cero, entonces se dice que es un \textbf{dominio de integridad}.
\end{definition}

\ejs $\Z, \Q, \R, \C, \F_p$

\begin{definition}{"Divide a"}{dividea}
Diremos que $a$ \textbf{divide a} $b$ en $R$ si $\exists c \in R$ tal que $b = a \cdot c$ y escribiremos $a | b$.
\end{definition}

\begin{definition}{Subanillo}{subanillo}
Diremos que $S \subset R$ es un \textbf{subanillo} si $(S, +, \cdot)$ es un anillo.
\end{definition}

\obs $S \subset R$ es un subanillo s y solo si:

\begin{enumerate}[label=\arabic*)]
	\item $S \neq \emptyset$
	\item $\forall a, b \in S, a + b \in S$
	\item $\forall a, b \in S, a \cdot b \in S$
	\item $1 \in S$
\end{enumerate}

\vspace{3mm}

\ej $\Z \subset \Q \subset \R \subset \C$

\begin{definition}{Menor subanillo que contiene a un elemento}{menoranillo}
Dado un anillo $R$ y un elemento $a$, podemos definir el \textbf{menor subanillo que contiene a $\mathbf{R}$ y al elemento $\mathbf{a}$} como $R[a] = \left \{ \displaystyle\sum r_i \cdot a^k, \forall r \in R; i, k \in \N \right \}$
\end{definition}

\ej $\Z[i] = \{a + bi, a, b \in \Z\} \subset \C$. Otra forma de ver este anillo es como la intersección de todos los subanillos de $\C$ que contienen a $\Z$ y a $i$.

\vspace{3mm}

\obs De la misma forma podemos definir el menor cuerpo que contiene a un elemento y que denotamos como $R(a)$.

\vspace{3mm}

\ej $\Q[\sqrt{2}] = \{a + b\sqrt{2}, \; a, b \in \Q\}$, $\Q(\sqrt{2}) = \left \{ \dfrac{a + b\sqrt{2}}{\underbrace{c + d\sqrt{2}}_{\neq 0}}, \; a, b, c, d \in \Q \right \}$, 

\end{document}