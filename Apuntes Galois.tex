\documentclass[10pt, a4paper]{article}
\usepackage[utf8]{inputenc}
\usepackage{amsfonts}
\usepackage{amsmath}

\usepackage{tcolorbox}
\usepackage{vmargin}
\usepackage{mathtools}
\usepackage{enumitem}
\usepackage{hyperref}
\tcbuselibrary{theorems}

\newcommand{\R}{\mathbb{R}}
\newcommand{\N}{\mathbb{N}}
\newcommand{\Z}{\mathbb{Z}}
\newcommand{\C}{\mathbb{C}}
\newcommand{\Q}{\mathbb{Q}}
\newcommand{\F}{\mathbb{F}}
\newcommand{\obs}[1][\!\!]{ \underline{Observación #1:} }
\newcommand{\ej}{\underline{Ejemplo:} }
\newcommand{\ejs}{\underline{Ejemplos:} }
\newcommand{\nota}{\underline{Notación:} }
\newcommand{\demo}{\underline{Demostración:} }

\tcbset{
	defstyle/.style={colback=blue!5, colframe=blue!80!black, before={\vspace{5mm}}, after={\vspace{5mm}}},, before={\vspace{5mm}}
	theostyle/.style={colback=red!5, colframe=red!80!black, before={\vspace{5mm}}, after={\vspace{5mm}}},
	propstyle/.style={colback=green!5, colframe=green!60!black, before={\vspace{5mm}}, after={\vspace{5mm}}}
}

\newtcbtheorem[number within=section]{definition}{Definición}{defstyle}{def}
\newtcbtheorem[number within=section]{theorem}{Teorema}{theostyle}{theo}
\newtcbtheorem[number within=section]{proposition}{Proposición}{propstyle}{prop}

\setlength{\parindent}{0pt}
\setmargins{2.5cm}       % margen izquierdo
{1.5cm}                        % margen superior
{16cm}                      % anchura del texto
{23.42cm}                    % altura del texto
{10pt}                           % altura de los encabezados
{1cm}                           % espacio entre el texto y los encabezados
{0pt}                             % altura del pie de página
{2cm}

\begin{document}

\title{Teoría de Galois}
\author{Carlos Gómez-Lobo}
\date{}
\maketitle

\section{Repaso de anillos}

A continuación vamos a repasar algunos conceptos sobre anillos y especialmente anillos de polinomios, empezando por la definición de anillo.

\begin{definition}{Anillo}{anillo}
Un \textbf{anillo} es un conjunto no vacío dotado de dos operaciones, que denotaremos como suma (=) y multiplicación (*) y que cumplen las siguientes propiedades:
\begin{itemize}
	\item $(R, +)$: grupo abeliano
	\item $(R, \cdot)$: operación binaria interna y cumple la propiedad asociativa
\end{itemize}
Si además $(R, \cdot)$ tiene identidad, es decir, existe un elemento $e \in R$ tal que $e \cdot r = r \cdot e = r \; \forall r \in R$, diremos que $R$ es un anillo con unidad y si además es abeliano, entonces será un anillo conmutativo.
\end{definition}

A nosotros en esta asignatura nos interesarán especialmente estos últimos y nos referiremos a estos simplemente como anillos sin especificar que son conmutativos y sin unidad.

\vspace{3mm}

\underline{Ejemplos:} $\Z, \Z_n, \R, \C, \Q, M_n(\R)$ (no conmutativo), etc.

\vspace{3mm}

\underline{Notación:} \begin{itemize}
	\item 0 para el elemento neutro de la suma
	\item $-a$ para el elemento inverso aditivo (opuesto).
	\item 1 para el elemento neutro de la multiplicación
	\item $a^{-1}$ para el inverso multiplicativo, si existe
	\item $na = \underbrace{a + ... + a}_{n veces}$
	
	\vspace{-3mm}
	
	\item $a^n = \underbrace{a \cdot ... \cdot a}_{n veces}$
\end{itemize}

\begin{definition}{Cuerpo}{cuerpo}
Un anillo $(R, +, -)$ es un \textbf{cuerpo} si $(R^* = R \backslash \{0\}, \cdot)$ es un grupo abeliano.
\end{definition}

\underline{Ejemplos:} $\Q, \R, \C, \Z_p$ p primo, etc.

\vspace{3mm}

\underline{Notación:} $\; \Z_n  \left \{
\begin{matrix*}[l]
\textrm{grupo aditivo} \rightarrow \C_n \\
\textrm{anillo} \rightarrow \Z_n \\
\textrm{cuerpo} \rightarrow \mathbb{F}_n \textrm{(n primo)}
\end{matrix*} \right .$

\begin{definition}{Divisor de cero}{div_cero}
Sea $R$ un anillo. Diremos que un elemento $a \in R, \; a \neq 0$ es un \textbf{divisor de cero} si $\exists b \in R, \; b \neq 0$ tal que $a \cdot b = 0$.
\end{definition}

\ej En $\Z_6: \bar{2}, \bar{3} \neq \bar{0}$ y $\bar{2} \cdot \bar{3} = 0$.

\begin{definition}{Dominio de integridad}{DI}
Sea $R$ un anillo, si $R$ no tiene divisores de cero, entonces se dice que es un \textbf{dominio de integridad}.
\end{definition}

\ejs $\Z, \Q, \R, \C, \F_p$

\begin{definition}{"Divide a"}{dividea}
Diremos que $a$ \textbf{divide a} $b$ en $R$ si $\exists c \in R$ tal que $b = a \cdot c$ y escribiremos $a | b$.
\end{definition}

\subsection{Subanillos}

\vspace{3mm}

\begin{definition}{Subanillo}{subanillo}
Diremos que $S \subset R$ es un \textbf{subanillo} si $(S, +, \cdot)$ es un anillo.
\end{definition}

\obs $S \subset R$ es un subanillo s y solo si:

\begin{enumerate}[label=\arabic*)]
	\item $S \neq \emptyset$
	\item $\forall a, b \in S, a + b \in S$
	\item $\forall a, b \in S, a \cdot b \in S$
	\item $1 \in S$
\end{enumerate}

\vspace{3mm}

\ej $\Z \subset \Q \subset \R \subset \C$

\begin{definition}{Menor subanillo que contiene a un elemento}{menoranillo}
Dado un anillo $R$ y un elemento $a$, podemos definir el \textbf{menor subanillo que contiene a $\mathbf{R}$ y al elemento $\mathbf{a}$} como $R[a] = \left \{ \displaystyle\sum r_i \cdot a^k, \forall r \in R; i, k \in \N \right \}$
\end{definition}

\ej $\Z[i] = \{a + bi, a, b \in \Z\} \subset \C$. Otra forma de ver este anillo es como la intersección de todos los subanillos de $\C$ que contienen a $\Z$ y a $i$.

\vspace{3mm}

\obs De la misma forma podemos definir el menor cuerpo que contiene a un elemento y que denotamos como $R(a)$.

\vspace{3mm}

\ej $\Q[\sqrt{2}] = \{a + b\sqrt{2}, \; a, b \in \Q\}$, $\Q(\sqrt{2}) = \bigg \{ \dfrac{a + b\sqrt{2}}{\underbrace{c + d\sqrt{2}}_{\neq 0}}, \; a, b, c, d \in \Q \bigg \}$, $\Q \subset \Q[\sqrt{2}] \subset \Q(\sqrt{2}) \subset \R$

\subsection{Anillos de polinomios}

\vspace{3mm}

\begin{definition}{Anillo de polinomios}{anillopoli}
Sea $R$ un anillo, llamaremos a $R[x]$ al \textbf{anillo de polinomios con coeficientes en $\mathbf{R}$} y que será de la forma $R[x] = \left \{ \displaystyle\sum_{k = 0}^{n} r_k x^k, \; \forall r \in R \right \}$.
\end{definition}

\ejs $\C[x], \R[x], \Q[x], \Z[x]$, etc.

\begin{definition}{Coeficiente director}{coef_dir}
El \textbf{coeficiente director} de un polinomio es el coeficiente distinto de 0 que multiplica a la x de mayor grado.
\end{definition}

\nota Grado de $p(x) := deg(p(x))$

\vspace{3mm}

\begin{proposition}{}{grado_prod}
El grado del producto de dos polinomios puede tener distintos valores en función de si el anillo sobre el que se construye es o no un \hyperref[def:DI]{DI}:
\[
deg(p(x) \cdot q(x)) = \left \{
\begin{matrix*}[l]
deg(p(x)) + deg(q(x)) \text{ si $R$ es dominio de integridad} \\
\leq deg(p(x)) + deg(q(x)) \text{ si no lo es}
\end{matrix*} \right .
\]
\end{proposition}

\demo Obvio.

\vspace{3mm}

\ej $\Z_4, \left .
\begin{matrix*}[l]
p(x) = 2x + 1 \\
q(x) = 2x
\end{matrix*} \right \}
deg(p(x) \cdot q(x) = 1 < 2$

\vspace{3mm}

\begin{proposition}{}{grado_DI}
 Sea $R$ un \hyperref[def:cuerpo]{cuerpo}, entonces $R$ es siempre \hyperref[def:DI]{dominio de integridad} y para cualesquiera polinomios de $R[x]$ se cumple que $deg(p(x)) \cdot deg(q(x)) = deg(p(x)) + deg(q(x))$.	
\end{proposition} 
 
\demo Para demostrar que un cuerpo siempre es un \hyperref[def:DI]{DI} vamos a ver por reducción al absurdo que todo elemento de un anillo que tenga inverso multiplicativo no es \hyperref[def:div_cero]{divisor de cero}.

Suponemos que $r \neq 0 \in R$ es \hyperref[def:div_cero]{divisor de cero}, es decir, $\exists r^{-1}$ tal que $r' \neq 0, r \cdot r' = 0$. Ahora suponemos además que $r$ es invertible, es decir, $\exists r^{-1}$ tal que $r \cdot r^{-1} = 1$. Entonces $r \cdot r^{-1} = 1 \implies (r' \cdot r) \cdot r^{-1} = b \implies 0 = b$. Contradicción.

De la misma forma se puede ver que una unidad no puede ser un \hyperref[def:div_cero]{divisor de cero} y como en un cuerpo todos sus elementos son unidades, no hay ningún \hyperref[def:div_cero]{divisor de cero} y por tanto es un \hyperref[def:DI]{dominio de integridad}.

Por esto y por la proposición \ref{prop:grado_prod}, queda demostrado.

\begin{proposition}{}{apoli_nocuerpo}
Sea $K$ un cuerpo, entonces el anillo de polinomios asociado a $K, \; K[x]$  \textbf{no} es un cuerpo y sus únicos elementos invertibles son los pertecientes al cuerpo $K$ no nulos.
\end{proposition}

\demo Sea $p(x) \in K[x], \; p(x) \neq 0$ invertible en $K[x]$. Entonces $p(x) \cdot p^{-1}(x) = 1, \; deg(1) = 0$ y como por la proposición \ref{prop:grado_prod}, $deg(p(x) \cdot p^{-1}(x)) \leq deg(p(x)) + deg(p^{-1}(x))$, se tiene que $deg(p(x)) = deg(p^{-1}(x)) = 0$, por lo que los únicos elementos invertibles en $K[x]$ son los de grado 0, que son los no nulos que pertenecen a $K$. Entonces, puesto que no todos los elementos de $K[x]$ son invertibles, $K[x]$ no es un cuerpo.

\vspace{3mm}

\obs El menor cuerpo que contiene a x y a los elementos de $K$ es \\ $K(x) = \left \{ \dfrac{p(x)}{q(x)} : p(x), q(x) \in K[x], q(x) \neq 0 \right \}$, teniendo en cuenta que $\dfrac{p(x)}{q(x)} = \dfrac{p'(x)}{q'(x)} \iff p(x) q'(x) = p'(x) q(x)$.

\begin{definition}{Polinomio mónico}{poli_mon}
Un \textbf{polinomio mónico} es aquel cuyo \hyperref[def:coef_dir]{coeficiente director} es 1.
\end{definition}

\subsection{Ideales en un anillo}

\vspace{3mm}

\begin{definition}{Ideal}{ideal}
Sea $R$ un anillo. Un \textbf{ideal} en $R$ es un subconjunto no vacío $I \subset R$ tal que:
\begin{enumerate}[label=\roman*)]
\item $(I, +)$ es un subgrupo de $R$
\end{enumerate}
\end{definition}

\end{document}